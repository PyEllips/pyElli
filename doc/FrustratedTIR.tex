% Encoding: utf-8

\chapterdoc{Example of the frustrated total internal reflection}
\chapterauthor{Olivier Castany, Céline Molinaro}

Frustrated total internal reflection is used as a validation example. 
Analytical and numerical results are compared.

\section{Presentation}

We consider the situation on figure~\ref{fig:FTIR}, where two half-spaces with indices $n_f$ and $n_b$ are separated by a medium of index $n_s$ and thickness $d$.
The three media are assumed to be lossless.
The incoming plane wave defines vector $k_x$ throughout the structure.
The reduced wave vector is $K_x=k_x/k_0=n_f \sin(\phi_i)$.
The scalar Helmholtz equation 
$(\Delta+k_0^2 \varepsilon)\{E,H\}=0$, 
holds separatley inside the three media, and implies that there are at most two waves in each medium, with wave vector $\pm k_z$, given by 
$k_z^2 = k_0^2 n^2 - k_x^2$.

\begin{figure}[!h]
\includegraphics[width=\columnwidth]{fig/FrustratedTIR}
\caption{\label{fig:FTIR}Frustrated total internal reflection with input and output plane waves. }
\end{figure}

We consider the separation medium.
Since $k_x$ is real, $k_z$ is either purely real or purely complex.
The first case happens for incidence angles $\phi_i$ smaller than the critical angle $\phi_{ic}$, given by $\sin(\phi_{ic}) = n_s/n_f$.
In this case, two plane waves are present, with $\pm k_z$ wavenumber.
The second case corresponds to the total internal reflexion in the front half-space medium and these is an evanescent wave in the separation medium.
If the third material is close enough, a plane wave is transmitted from the evanescent wave.
This phenomenon is the frustrated total internal reflexion.
In this situation, two evanescent waves are present in the medium, with $\pm k_z''$ exponential evolution.

Due to the ($xz$) mirror symmetry, $s$ and $p$ modes can be considered separately. 
For $s$ polarization, we have
$$
\vec E = E(x,z)\ \vec y
\quad\mathrm{and}\quad
\vec H = \frac{1}{i k_0} 
\begin{pmatrix}
-\partial_z E \\
\phantom{-} 0 \\
\phantom{-} \partial_x E
\end{pmatrix}.
$$
For $p$ polarization, we have
$$
\vec H = H(x,z)\ \vec y
\quad\mathrm{and}\quad
\vec E = \frac{i}{k_0 \varepsilon} 
\begin{pmatrix}
-\partial_z H \\
\phantom{-} 0 \\
\phantom{-} \partial_x H
\end{pmatrix}.
$$
This duality implies that expressions for $p$ polarization relate naturally to $H$. 
However, transmission and reflection coefficients are always defined with respect to $E$,
$$
t=\frac{E_t}{E_i}
\quad\textrm{and}\quad
r=\frac{E_r}{E_i}
\textrm{, for both $s$ and $p$ polarizations.}
$$

In the next sections, we will study the anatomy of the waves in detail. 
For that purpose, the Poynting vector gives useful physical insight.
In Gaussian units, the time average of the Poynting vector is
$$\langle \vec{\Pi} \rangle_t  =  
\frac{c}{8\pi} \Re\left(\vec{E}\times \vec{H}^*\right).$$

\section{Anatomy of a single wave}

If we consider an $s$-polarized wave defined by 
$$
\vec E = E(x,y)\ \vec y
\quad\mathrm{with}\quad
E(x,y) = E_0\ e^{-i\omega t + i(k_x x + k_z z)},$$
we deduce
$\quad
\vec H = E(x,y)
\begin{pmatrix}
- K_z\\
\phantom{-} 0\\
\phantom{-} K_x
\end{pmatrix},
$
$$
\vec H^* = E_0^*\ e^{i\omega t - i(k_x x + k_z^* z)}
\begin{pmatrix}
- K_z^*\\
\phantom{-} 0\\
\phantom{-} K_x
\end{pmatrix},
$$
\begin{equation}\label{eq:Poynting-s}
\mathrm{and}\quad
\langle \vec{\Pi} \rangle_t  = \frac{c}{8\pi} 
|E_0|^2 e^{-2 k_z'' z}
\begin{pmatrix}
K_x\\
0\\
K_z'
\end{pmatrix}.
\end{equation}
If we consider a $p$-polarized wave defined by 
$$
\vec H = H(x,y)\ \vec y
\quad\mathrm{with}\quad
H(x,y) = H_0\ e^{-i\omega t + i(k_x x + k_z z)},$$
\begin{flalign}
\textrm{we deduce}
\quad
\vec E = \displaystyle \frac{H(x,y)}{\varepsilon}
\begin{pmatrix}
\phantom{-} K_z\\
\phantom{-} 0\\
- K_x
\end{pmatrix},&&
\label{eq:p-wave-single}
\end{flalign}
\begin{equation*}
\vec H^* = H_0^*\ e^{i\omega t - i(k_x x + k_z^* z)}\ \vec y,
\end{equation*}
\begin{equation}\label{eq:Poynting-p}
\mathrm{and}\quad
\langle \vec{\Pi} \rangle_t  = \frac{c}{8\pi} 
\frac{|H_0|^2}{\varepsilon} e^{-2 k_z'' z}
\begin{pmatrix}
K_x\\
0\\
K_z'
\end{pmatrix}.
\end{equation}

In these expressions, we defined $k_z = k_z' + i k_z''$.
We observe that the Poynting vector is parallel to the real part of the wave vector.
If $k_z$ is real ($k_z''=0$), the wave amplitude is constant along $z$.
If $k_z$ is purely complex ($k_z'=0$), there is no energy flow in the $z$ direction, and the wave decays exponentially in the $z$ direction.
Figure~\ref{fig:Plane_wave_real_complex} represents the two cases.

\begin{figure}[h!]
\includegraphics[width=\columnwidth]{fig/Plane_wave_real_complex}
\caption{\label{fig:Plane_wave_real_complex}Anatomy of a homogeneous plane wave and an evanescent wave. The arrows show the real part of the wave vector. The thickness of the arrow indicates the intensity of the wave.}
\end{figure}


\section{Reflection on an interface: fields and Poynting vector}

We consider an incident wave partially reflected at $z=0$ by a structure that does not affect the parity of the wave ($s$ or $p$ polarization).
The complex reflection coefficients are called $r_s$ and $r_p$.
The waves may either be plane or evanescent waves.
The incident and reflected waves are named with ``$+$'' and ``$-$'' subscripts.
The total field is $\vec E = \vec E^+ + \vec E^-$.
%
For the $s$ polarisation, we consider
$$
\left\{
\begin{array}{l}
\vec E^+ = E^+\ \vec y \\
\vec E^- = E^-\ \vec y
\end{array}
\right.
\quad\mathrm{with}\quad
\left\{
\begin{array}{l}
E^+ = E_0^+ e^{-i\omega t + i(k_x x + k_z z)}  \\
E^- = E_0^- e^{-i\omega t + i(k_x x - k_z z)}.
\end{array}
\right.
$$
The magnetic excitation is 
$$
\vec H = \vec H^+ + \vec H^- = 
E^+ 
\begin{pmatrix}
- K_z \\
\phantom{-} 0 \\
\phantom{-} K_x 
\end{pmatrix}
+
E^-
\begin{pmatrix}
K_z \\
0 \\
K_x 
\end{pmatrix}.
$$
The amplitudes of the incident and reflected waves are connected by $r_s = E_0^- / E_0^+$ and the Poynting vector is
\begin{align*}
\langle \vec{\Pi} \rangle_t = 
\frac{c}{8\pi} |E_0^+|^2 \left\{
e^{-2 k_z'' z} 
\begin{pmatrix}
K_x \\
0 \\
K_z'
\end{pmatrix}
+ 
|r_s|^2
e^{2 k_z'' z}
\begin{pmatrix}
\phantom{-} K_x \\
\phantom{-} 0 \\
-K_z'
\end{pmatrix}
+ \right.\\
\left. +
2\ |r_s|\ 
\begin{pmatrix}
K_x \cos(\theta_{s} - 2 k_z' z)\\
0 \\
K_z'' \sin(\theta_{s} - 2 k_z' z)
\end{pmatrix}
\right\}
\end{align*}
where we defined $r_s = |r_s|\ e^{i\theta_{s}}$.
The first and second terms correspond to the incident and reflected waves.
The third term arises from the interference of the two waves.
If $K_z$ is real, the expression in curly braces becomes
$$
\begin{pmatrix}
K_x \\
0 \\
K_z
\end{pmatrix}
+ 
|r_s|^2
\begin{pmatrix}
\phantom{-} K_x \\
\phantom{-} 0 \\
- K_z
\end{pmatrix}
+ 2\ |r_s|\ 
\begin{pmatrix}
K_x \cos(\theta_{s} - 2 k_z z)\\
0 \\
0 \\
\end{pmatrix}.
$$
If $K_z$ is purely complex, it becomes
$$
e^{-2 k_z'' z} 
\begin{pmatrix}
K_x \\
0 \\
0
\end{pmatrix}
+ 
|r_s|^2
e^{2 k_z'' z}
\begin{pmatrix}
K_x \\
0 \\
0
\end{pmatrix}
+ 2\ 
\begin{pmatrix}
K_x r_s'\\
0 \\
K_z'' r_s''
\end{pmatrix},
$$
which exhibits an energy flow in the $z$ direction, proportionnal to $K_z'' r_s''$.

For $p$ polarisation, we consider 
$$
\left\{
\begin{array}{l}
\vec H^+ = H^+\ \vec y \\
\vec H^- = H^-\ \vec y
\end{array}
\right.
\quad\mathrm{with}\quad
\left\{
\begin{array}{l}
H^+ = H_0^+ e^{-i\omega t + i(k_x x + k_z z)}  \\
H^- = H_0^- e^{-i\omega t + i(k_x x - k_z z)}.
\end{array}
\right.
$$
The electric field is
\begin{equation}
\vec E = \vec E^+ + \vec E^- = 
\frac{H^+}{\varepsilon}
\begin{pmatrix}
\phantom{-} K_z \\
\phantom{-} 0 \\
- K_x 
\end{pmatrix}
+
\frac{H^-}{\varepsilon}
\begin{pmatrix}
- K_z \\
\phantom{-} 0 \\
- K_x 
\end{pmatrix}.
\label{eq:p-wave-superposition}
\end{equation}
The amplitudes of the incident and reflected waves are connected by $r_p = E_p^- / E_p^+ = H_0^- / H_0^+$.
The Poynting vector has the same expression as for the $s$ polarization, with $r_s$ replaced by $r_p = |r_p|\ e^{i\theta_{p}}$ and $|E_0^+|^2$ replaced by $|H_0^+|^2/\varepsilon$,
\begin{align*}
\langle \vec{\Pi} \rangle_t = 
\frac{c}{8\pi} \frac{|H_0^+|^2}{\varepsilon} \left\{
e^{-2 k_z'' z} 
\begin{pmatrix}
K_x \\
0 \\
K_z'
\end{pmatrix}
+ 
|r_p|^2
e^{2 k_z'' z}
\begin{pmatrix}
\phantom{-} K_x \\
\phantom{-} 0 \\
-K_z'
\end{pmatrix}
+ \right.\\
\left. +
2\ |r_p|\ 
\begin{pmatrix}
K_x \cos(\theta_{p} - 2 k_z' z)\\
0 \\
K_z'' \sin(\theta_{p} - 2 k_z' z)
\end{pmatrix}
\right\}.
\end{align*}
If $K_z$ is purely complex, there is an energy flow in the $z$ direction, proportionnal to $K_z'' r_p''$.
Figure~\ref{fig:Wave_evanescent_poynting} represents the variation of the Poynting vector along $z$, in the case of evanescent waves.

\begin{figure}[!h]
\includegraphics[width=\columnwidth]{fig/Wave_evanescent_poynting}
\caption{\label{fig:Wave_evanescent_poynting}Refection of an evanescent wave on an interface ($k_z$ is purely complex and $r''\ge 0$). The Poynting vector is decomposed in different terms.}
\end{figure}

\section{Expression of the reflexion coefficients for an interface between two media}

We consider the reflexion on an interface between two media ``1'' and ``2'', with light coming from medium ``1''.
The reflection and transmission coefficients for the $s$ polarization are\cite{Wikipedia_Fresnel}
$$
r_s = \frac{k_{z1}-k_{z2}}{k_{z1}+k_{z2}} 
\quad\textrm{and}\quad
t_s = 1 + r_s = \frac{2\ k_{z1}}{k_{z1}+k_{z2}}.
$$
The expressions are valid for complex wave vectors and we deduce
$$
r_s' = \frac{|k_{z1}|^2 - |k_{z2}|^2}{|k_{z1} + k_{z2}|^2}
\quad\textrm{and}\quad
r_s'' = \frac{2 (k_{z1}'' k_{z2}' - k_{z1}' k_{z2}'')}{|k_{z1} + k_{z2}|^2}.
$$
In the case of an evanescent wave in region~1, the wave number is $k_{z1} = i k_{z1}''$ and we deduce
$$
r_s' = \frac{k_{z1}''^2 - |k_{z2}|^2}{|ik_{z1}'' + k_{z2}|^2}
\quad\textrm{and}\quad
r_s'' = \frac{2 k_{z1}'' k_{z2}'}{|ik_{z1}'' + k_{z2}|^2}
\ .
$$
The last expression shows that $r_s'' \ge 0$, which implies that the energy flow is directed to the right, as expected.

For $p$ polarization, the coefficients for the magnetic field are
$r^H_p = H^-_{1p} / H^+_{1p}$ and $t^H_p = H^+_{2p} / H^+_{1p}$ with
$$
r^H_p=\frac{\epsilon_2k_{z1}-\epsilon_1k_{z2}}{\epsilon_2k_{z1}+\epsilon_1k_{z2}}
\quad\textrm{and}\quad
t^H_p = 1+r^H_p = \frac{2\epsilon_2 k_{z1}}{\epsilon_2k_{z1}+\epsilon_1k_{z2}}
.
$$
The direction and value of the electric field is deduced from equations~\ref{eq:p-wave-single} and \ref{eq:p-wave-superposition}, which leads to $r_p=-r^H_p$ and $t_p = n_1/n_2 \times t^H_p$ and implies \cite{Wikipedia_Fresnel}
$$
r_p=
\frac{\epsilon_1k_{z2} - \epsilon_2k_{z1}}{\epsilon_2k_{z1}+\epsilon_1k_{z2}}
\quad\textrm{and}\quad 
t_p = 
\frac{2 n_1 n_2 k_{z1}}{\epsilon_2k_{z1}+\epsilon_1k_{z2}}.
$$
We deduce
$ \displaystyle
r_p' = \frac{|\epsilon_1 k_{z2}|^2 - |\epsilon_2 k_{z1}|^2}{|\epsilon_2 k_{z1} + \epsilon_1 k_{z2}|^2}$ 
and
$$
r_p'' = \frac{2 \epsilon_1\epsilon_2(k_{z1}'' k_{z2}' - k_{z1}' k_{z2}'')}{|\epsilon_2 k_{z1} + \epsilon_1 k_{z2}|^2}.
$$
In the case of an evanescent wave in region~1, the wave number is $k_{z1} = i k_{z1}''$ and we deduce
$$
r_p' = \frac{|\epsilon_1 k_{z2}|^2 - \epsilon_2^2 k_{z1}''^2}{|\epsilon_2 i k_{z1}'' + \epsilon_1 k_{z2}|^2}
\quad\textrm{and}\quad
r_p'' =  \frac{2 \epsilon_1\epsilon_2 k_{z1}'' k_{z2}'}{|\epsilon_2 i k_{z1}'' + \epsilon_1 k_{z2}|^2}
.
$$
The last expression shows that $r_p''\ge 0$, which implies that the energy flow is directed to the right, as expected.

The power flow along the $z$ direction is deduced from equations~\ref{eq:Poynting-s} and \ref{eq:Poynting-p}, leading to the expressions for the power coefficients for both polarizations,
$$
R = \frac{\langle \Pi_{z1}^- \rangle_t}{\langle \Pi_{z1}^+ \rangle_t}
= |r|^2
\quad\textrm{and}\quad
T = \frac{\langle \Pi_{z2}^+ \rangle_t}{\langle \Pi_{z1}^+ \rangle_t} 
= \frac{k_{z2}'}{k_{z1}'} \times |t|^2.
$$
We verify that $R+T=1$ for both polarizations when the materials are lossless%
\footnote{The relation only has a meaning when $k_{z1}$ is real. The demonstration uses the fact that $k_{z2}$ is either real or purely complex. The two cases are considered separately and both verify $R+T=1$.}.
%
Also, if we consider the coefficients for the reverse directions, we show that for both polarizations we have
$r_{12} = - r_{21}$,
$t_{12}/k_2 = t_{21}/k_1$
and
$t_{12} t_{21} + r_{12}^2 = 1$
.


\section{Application to the frustrated total internal reflection}

We consider waves $E_i$ and $E_r$ in the incident medium, $E^+$ and $E^-$ in the separation medium, and $E_t$ in the back half-space.
The relations at the interfaces can be written in the same fashion for both polarizations. 
At the $z=0$ interface, we have
\begin{equation*}
\left\{
\begin{array}{lllll}\label{eq:continuity0}
E^+(0)  & = & r_{fs}\ E^-(0)  & + & t_{sf}\ E_i(0)    \\
E_r(0)  & = & r_{sf}\ E_i(0)  & + & t_{fs}\ E^-(0)
\end{array}\right.
\end{equation*}
and at the $z=d$ interface, we have
\begin{equation*}
\left\{
\begin{array}{lll}\label{eq:continuityd}
E_t(d)  & = & t_{bs}\ E^+(d)  \\
E^-(d)  & = & r_{bs}\ E^+(d).
\end{array}\right.
\end{equation*}
Propagation between planes $z=0$ and $z=d$ implies
\begin{equation*}
\left\{
\begin{array}{lll}\label{eq:0tod}
E^+(d) & = E^+(0)\ e^{ik_z d}\\
E^-(0) & = E^-(d)\ e^{ik_z d} \ ,
\end{array}\right.
\end{equation*}
where $k_z$ is the wavenumber in the separation medium.
From these equations, we get%
\footnote{We used the relation $t_{fs} t_{sf} - r_{fs} r_{sf} = 1$ for simplifying the expression of the coefficient $r$.}
\begin{equation*}
E^+(0) / E_i(0) =\frac{t_{sf}}{1 - r_{fs}r_{bs}e^{i2k_z d}}
\end{equation*}
%
\begin{equation*}
E^-(0) / E_i(0) =\frac{r_{bs} t_{sf} e^{i2k_z d}}{1-r_{fs}r_{bs}e^{i2k_z d}}
\end{equation*}
%
\begin{equation*}\label{eq:reflectivewave}
r = E_r(0) / E_i(0) =
\frac{r_{sf} + r_{bs} e^{i2k_z d}}{1-r_{fs}r_{bs}e^{i2k_z d}}
\end{equation*}
%
\begin{equation*}\label{eq:transmittedwave}
t = E_t(d) / E_i(0) =
\frac{t_{bs} t_{sf} e^{ik_z d}}{1 - r_{fs}r_{bs}e^{i2k_z d}}
\end{equation*}
The power reflection and transmission coefficient are
$$
R = |r|^2 
\quad\textrm{and}\quad
T = \frac{\langle \Pi_{zt} \rangle_t}{\langle \Pi_{zi} \rangle_t} 
= \frac{k_{zb}'}{k_{zf}'} \ |t|^2
% = \frac{n_b\cos(\Phi _t)}{n_f\cos(\Phi _i)}\ |t|^2
.
$$
We verify that $R+T=1$ when the materials are lossless.
To demonstrate this, different cases are separated, depending on the type of wave in the regions.
In the case when there are plane waves in both half-spaces, we show that 
$
k_{zb}'/k_{zf}'\ |t_{bs}t_{sf}|^2 \exp(-2 k_z'' d)
= |t_{fs} t_{sf} t_{bs} t_{sb}| \exp(-2 k_z'' d)
$.
When there is also a plane wave in the separation region, the reflection coefficients are real, and the term simplifies to 
$(1-r_{fs}^2)(1-r_{bs}^2)$.
When there is an evanescent wave in the separation medium, the reflection coefficients are complex with unitary modulus, and the term simplifies to 
$4\ r_{bs}'' r_{fs}'' \exp(-2 k_z'' d)$.
In both cases, putting this term in the expansion of $R+T$ leads to the result
$R+T=1$.

